
%% bare_jrnl_compsoc.tex
%% V1.4a
%% 2014/09/17
%% by Michael Shell
%% See:
%% http://www.michaelshell.org/
%% for current contact information.
%%
%% This is a skeleton file demonstrating the use of IEEEtran.cls
%% (requires IEEEtran.cls version 1.8a or later) with an IEEE
%% Computer Society journal paper.
%%
%% Support sites:
%% http://www.michaelshell.org/tex/ieeetran/
%% http://www.ctan.org/tex-archive/macros/latex/contrib/IEEEtran/
%% and
%% http://www.ieee.org/

%%*************************************************************************
%% Legal Notice:
%% This code is offered as-is without any warranty either expressed or
%% implied; without even the implied warranty of MERCHANTABILITY or
%% FITNESS FOR A PARTICULAR PURPOSE! 
%% User assumes all risk.
%% In no event shall IEEE or any contributor to this code be liable for
%% any damages or losses, including, but not limited to, incidental,
%% consequential, or any other damages, resulting from the use or misuse
%% of any information contained here.
%%
%% All comments are the opinions of their respective authors and are not
%% necessarily endorsed by the IEEE.
%%
%% This work is distributed under the LaTeX Project Public License (LPPL)
%% ( http://www.latex-project.org/ ) version 1.3, and may be freely used,
%% distributed and modified. A copy of the LPPL, version 1.3, is included
%% in the base LaTeX documentation of all distributions of LaTeX released
%% 2003/12/01 or later.
%% Retain all contribution notices and credits.
%% ** Modified files should be clearly indicated as such, including  **
%% ** renaming them and changing author support contact information. **
%%
%% File list of work: IEEEtran.cls, IEEEtran_HOWTO.pdf, bare_adv.tex,
%%                    bare_conf.tex, bare_jrnl.tex, bare_conf_compsoc.tex,
%%                    bare_jrnl_compsoc.tex, bare_jrnl_transmag.tex
%%*************************************************************************


% *** Authors should verify (and, if needed, correct) their LaTeX system  ***
% *** with the testflow diagnostic prior to trusting their LaTeX platform ***
% *** with production work. IEEE's font choices and paper sizes can       ***
% *** trigger bugs that do not appear when using other class files.       ***                          ***
% The testflow support page is at:
% http://www.michaelshell.org/tex/testflow/


\documentclass[10pt,journal,compsoc]{IEEEtran}
%
% If IEEEtran.cls has not been installed into the LaTeX system files,
% manually specify the path to it like:
% \documentclass[10pt,journal,compsoc]{../sty/IEEEtran}





% Some very useful LaTeX packages include:
% (uncomment the ones you want to load)


% *** MISC UTILITY PACKAGES ***
%
%\usepackage{ifpdf}
% Heiko Oberdiek's ifpdf.sty is very useful if you need conditional
% compilation based on whether the output is pdf or dvi.
% usage:
% \ifpdf
%   % pdf code
% \else
%   % dvi code
% \fi
% The latest version of ifpdf.sty can be obtained from:
% http://www.ctan.org/tex-archive/macros/latex/contrib/oberdiek/
% Also, note that IEEEtran.cls V1.7 and later provides a builtin
% \ifCLASSINFOpdf conditional that works the same way.
% When switching from latex to pdflatex and vice-versa, the compiler may
% have to be run twice to clear warning/error messages.






% *** CITATION PACKAGES ***
%
\ifCLASSOPTIONcompsoc
  % IEEE Computer Society needs nocompress option
  % requires cite.sty v4.0 or later (November 2003)
  \usepackage[nocompress]{cite}
\else
  % normal IEEE
  \usepackage{cite}
\fi
% cite.sty was written by Donald Arseneau
% V1.6 and later of IEEEtran pre-defines the format of the cite.sty package
% \cite{} output to follow that of IEEE. Loading the cite package will
% result in citation numbers being automatically sorted and properly
% "compressed/ranged". e.g., [1], [9], [2], [7], [5], [6] without using
% cite.sty will become [1], [2], [5]--[7], [9] using cite.sty. cite.sty's
% \cite will automatically add leading space, if needed. Use cite.sty's
% noadjust option (cite.sty V3.8 and later) if you want to turn this off
% such as if a citation ever needs to be enclosed in parenthesis.
% cite.sty is already installed on most LaTeX systems. Be sure and use
% version 5.0 (2009-03-20) and later if using hyperref.sty.
% The latest version can be obtained at:
% http://www.ctan.org/tex-archive/macros/latex/contrib/cite/
% The documentation is contained in the cite.sty file itself.
%
% Note that some packages require special options to format as the Computer
% Society requires. In particular, Computer Society  papers do not use
% compressed citation ranges as is done in typical IEEE papers
% (e.g., [1]-[4]). Instead, they list every citation separately in order
% (e.g., [1], [2], [3], [4]). To get the latter we need to load the cite
% package with the nocompress option which is supported by cite.sty v4.0
% and later. Note also the use of a CLASSOPTION conditional provided by
% IEEEtran.cls V1.7 and later.





% *** GRAPHICS RELATED PACKAGES ***
%
\ifCLASSINFOpdf
  % \usepackage[pdftex]{graphicx}
  % declare the path(s) where your graphic files are
  % \graphicspath{{../pdf/}{../jpeg/}}
  % and their extensions so you won't have to specify these with
  % every instance of \includegraphics
  % \DeclareGraphicsExtensions{.pdf,.jpeg,.png}
\else
  % or other class option (dvipsone, dvipdf, if not using dvips). graphicx
  % will default to the driver specified in the system graphics.cfg if no
  % driver is specified.
  % \usepackage[dvips]{graphicx}
  % declare the path(s) where your graphic files are
  % \graphicspath{{../eps/}}
  % and their extensions so you won't have to specify these with
  % every instance of \includegraphics
  % \DeclareGraphicsExtensions{.eps}
\fi
% graphicx was written by David Carlisle and Sebastian Rahtz. It is
% required if you want graphics, photos, etc. graphicx.sty is already
% installed on most LaTeX systems. The latest version and documentation
% can be obtained at: 
% http://www.ctan.org/tex-archive/macros/latex/required/graphics/
% Another good source of documentation is "Using Imported Graphics in
% LaTeX2e" by Keith Reckdahl which can be found at:
% http://www.ctan.org/tex-archive/info/epslatex/
%
% latex, and pdflatex in dvi mode, support graphics in encapsulated
% postscript (.eps) format. pdflatex in pdf mode supports graphics
% in .pdf, .jpeg, .png and .mps (metapost) formats. Users should ensure
% that all non-photo figures use a vector format (.eps, .pdf, .mps) and
% not a bitmapped formats (.jpeg, .png). IEEE frowns on bitmapped formats
% which can result in "jaggedy"/blurry rendering of lines and letters as
% well as large increases in file sizes.
%
% You can find documentation about the pdfTeX application at:
% http://www.tug.org/applications/pdftex






% *** MATH PACKAGES ***
%
%\usepackage[cmex10]{amsmath}
% A popular package from the American Mathematical Society that provides
% many useful and powerful commands for dealing with mathematics. If using
% it, be sure to load this package with the cmex10 option to ensure that
% only type 1 fonts will utilized at all point sizes. Without this option,
% it is possible that some math symbols, particularly those within
% footnotes, will be rendered in bitmap form which will result in a
% document that can not be IEEE Xplore compliant!
%
% Also, note that the amsmath package sets \interdisplaylinepenalty to 10000
% thus preventing page breaks from occurring within multiline equations. Use:
%\interdisplaylinepenalty=2500
% after loading amsmath to restore such page breaks as IEEEtran.cls normally
% does. amsmath.sty is already installed on most LaTeX systems. The latest
% version and documentation can be obtained at:
% http://www.ctan.org/tex-archive/macros/latex/required/amslatex/math/





% *** SPECIALIZED LIST PACKAGES ***
%
%\usepackage{algorithmic}
% algorithmic.sty was written by Peter Williams and Rogerio Brito.
% This package provides an algorithmic environment fo describing algorithms.
% You can use the algorithmic environment in-text or within a figure
% environment to provide for a floating algorithm. Do NOT use the algorithm
% floating environment provided by algorithm.sty (by the same authors) or
% algorithm2e.sty (by Christophe Fiorio) as IEEE does not use dedicated
% algorithm float types and packages that provide these will not provide
% correct IEEE style captions. The latest version and documentation of
% algorithmic.sty can be obtained at:
% http://www.ctan.org/tex-archive/macros/latex/contrib/algorithms/
% There is also a support site at:
% http://algorithms.berlios.de/index.html
% Also of interest may be the (relatively newer and more customizable)
% algorithmicx.sty package by Szasz Janos:
% http://www.ctan.org/tex-archive/macros/latex/contrib/algorithmicx/




% *** ALIGNMENT PACKAGES ***
%
%\usepackage{array}
% Frank Mittelbach's and David Carlisle's array.sty patches and improves
% the standard LaTeX2e array and tabular environments to provide better
% appearance and additional user controls. As the default LaTeX2e table
% generation code is lacking to the point of almost being broken with
% respect to the quality of the end results, all users are strongly
% advised to use an enhanced (at the very least that provided by array.sty)
% set of table tools. array.sty is already installed on most systems. The
% latest version and documentation can be obtained at:
% http://www.ctan.org/tex-archive/macros/latex/required/tools/


% IEEEtran contains the IEEEeqnarray family of commands that can be used to
% generate multiline equations as well as matrices, tables, etc., of high
% quality.




% *** SUBFIGURE PACKAGES ***
%\ifCLASSOPTIONcompsoc
%  \usepackage[caption=false,font=footnotesize,labelfont=sf,textfont=sf]{subfig}
%\else
%  \usepackage[caption=false,font=footnotesize]{subfig}
%\fi
% subfig.sty, written by Steven Douglas Cochran, is the modern replacement
% for subfigure.sty, the latter of which is no longer maintained and is
% incompatible with some LaTeX packages including fixltx2e. However,
% subfig.sty requires and automatically loads Axel Sommerfeldt's caption.sty
% which will override IEEEtran.cls' handling of captions and this will result
% in non-IEEE style figure/table captions. To prevent this problem, be sure
% and invoke subfig.sty's "caption=false" package option (available since
% subfig.sty version 1.3, 2005/06/28) as this is will preserve IEEEtran.cls
% handling of captions.
% Note that the Computer Society format requires a sans serif font rather
% than the serif font used in traditional IEEE formatting and thus the need
% to invoke different subfig.sty package options depending on whether
% compsoc mode has been enabled.
%
% The latest version and documentation of subfig.sty can be obtained at:
% http://www.ctan.org/tex-archive/macros/latex/contrib/subfig/




% *** FLOAT PACKAGES ***
%
%\usepackage{fixltx2e}
% fixltx2e, the successor to the earlier fix2col.sty, was written by
% Frank Mittelbach and David Carlisle. This package corrects a few problems
% in the LaTeX2e kernel, the most notable of which is that in current
% LaTeX2e releases, the ordering of single and double column floats is not
% guaranteed to be preserved. Thus, an unpatched LaTeX2e can allow a
% single column figure to be placed prior to an earlier double column
% figure. The latest version and documentation can be found at:
% http://www.ctan.org/tex-archive/macros/latex/base/


%\usepackage{stfloats}
% stfloats.sty was written by Sigitas Tolusis. This package gives LaTeX2e
% the ability to do double column floats at the bottom of the page as well
% as the top. (e.g., "\begin{figure*}[!b]" is not normally possible in
% LaTeX2e). It also provides a command:
%\fnbelowfloat
% to enable the placement of footnotes below bottom floats (the standard
% LaTeX2e kernel puts them above bottom floats). This is an invasive package
% which rewrites many portions of the LaTeX2e float routines. It may not work
% with other packages that modify the LaTeX2e float routines. The latest
% version and documentation can be obtained at:
% http://www.ctan.org/tex-archive/macros/latex/contrib/sttools/
% Do not use the stfloats baselinefloat ability as IEEE does not allow
% \baselineskip to stretch. Authors submitting work to the IEEE should note
% that IEEE rarely uses double column equations and that authors should try
% to avoid such use. Do not be tempted to use the cuted.sty or midfloat.sty
% packages (also by Sigitas Tolusis) as IEEE does not format its papers in
% such ways.
% Do not attempt to use stfloats with fixltx2e as they are incompatible.
% Instead, use Morten Hogholm'a dblfloatfix which combines the features
% of both fixltx2e and stfloats:
%
% \usepackage{dblfloatfix}
% The latest version can be found at:
% http://www.ctan.org/tex-archive/macros/latex/contrib/dblfloatfix/




%\ifCLASSOPTIONcaptionsoff
%  \usepackage[nomarkers]{endfloat}
% \let\MYoriglatexcaption\caption
% \renewcommand{\caption}[2][\relax]{\MYoriglatexcaption[#2]{#2}}
%\fi
% endfloat.sty was written by James Darrell McCauley, Jeff Goldberg and 
% Axel Sommerfeldt. This package may be useful when used in conjunction with 
% IEEEtran.cls'  captionsoff option. Some IEEE journals/societies require that
% submissions have lists of figures/tables at the end of the paper and that
% figures/tables without any captions are placed on a page by themselves at
% the end of the document. If needed, the draftcls IEEEtran class option or
% \CLASSINPUTbaselinestretch interface can be used to increase the line
% spacing as well. Be sure and use the nomarkers option of endfloat to
% prevent endfloat from "marking" where the figures would have been placed
% in the text. The two hack lines of code above are a slight modification of
% that suggested by in the endfloat docs (section 8.4.1) to ensure that
% the full captions always appear in the list of figures/tables - even if
% the user used the short optional argument of \caption[]{}.
% IEEE papers do not typically make use of \caption[]'s optional argument,
% so this should not be an issue. A similar trick can be used to disable
% captions of packages such as subfig.sty that lack options to turn off
% the subcaptions:
% For subfig.sty:
% \let\MYorigsubfloat\subfloat
% \renewcommand{\subfloat}[2][\relax]{\MYorigsubfloat[]{#2}}
% However, the above trick will not work if both optional arguments of
% the \subfloat command are used. Furthermore, there needs to be a
% description of each subfigure *somewhere* and endfloat does not add
% subfigure captions to its list of figures. Thus, the best approach is to
% avoid the use of subfigure captions (many IEEE journals avoid them anyway)
% and instead reference/explain all the subfigures within the main caption.
% The latest version of endfloat.sty and its documentation can obtained at:
% http://www.ctan.org/tex-archive/macros/latex/contrib/endfloat/
%
% The IEEEtran \ifCLASSOPTIONcaptionsoff conditional can also be used
% later in the document, say, to conditionally put the References on a 
% page by themselves.




% *** PDF, URL AND HYPERLINK PACKAGES ***
%
\usepackage{url}
% url.sty was written by Donald Arseneau. It provides better support for
% handling and breaking URLs. url.sty is already installed on most LaTeX
% systems. The latest version and documentation can be obtained at:
% http://www.ctan.org/tex-archive/macros/latex/contrib/url/
% Basically, \url{my_url_here}.

\usepackage[utf8]{inputenc}
\usepackage[T1]{fontenc}
\usepackage[spanish]{babel}
\usepackage{graphicx}
\graphicspath{ {images/} }



% *** Do not adjust lengths that control margins, column widths, etc. ***
% *** Do not use packages that alter fonts (such as pslatex).         ***
% There should be no need to do such things with IEEEtran.cls V1.6 and later.
% (Unless specifically asked to do so by the journal or conference you plan
% to submit to, of course. )


% correct bad hyphenation here
\hyphenation{op-tical net-works semi-conduc-tor}


\begin{document}
%
% paper title
% Titles are generally capitalized except for words such as a, an, and, as,
% at, but, by, for, in, nor, of, on, or, the, to and up, which are usually
% not capitalized unless they are the first or last word of the title.
% Linebreaks \\ can be used within to get better formatting as desired.
% Do not put math or special symbols in the title.
\title{MPLS-DWDM}
%
%
% author names and IEEE memberships
% note positions of commas and nonbreaking spaces ( ~ ) LaTeX will not break
% a structure at a ~ so this keeps an author's name from being broken across
% two lines.
% use \thanks{} to gain access to the first footnote area
% a separate \thanks must be used for each paragraph as LaTeX2e's \thanks
% was not built to handle multiple paragraphs
%
%
%\IEEEcompsocitemizethanks is a special \thanks that produces the bulleted
% lists the Computer Society journals use for "first footnote" author
% affiliations. Use \IEEEcompsocthanksitem which works much like \item
% for each affiliation group. When not in compsoc mode,
% \IEEEcompsocitemizethanks becomes like \thanks and
% \IEEEcompsocthanksitem becomes a line break with idention. This
% facilitates dual compilation, although admittedly the differences in the
% desired content of \author between the different types of papers makes a
% one-size-fits-all approach a daunting prospect. For instance, compsoc 
% journal papers have the author affiliations above the "Manuscript
% received ..."  text while in non-compsoc journals this is reversed. Sigh.



% note the % following the last \IEEEmembership and also \thanks - 
% these prevent an unwanted space from occurring between the last author name
% and the end of the author line. i.e., if you had this:
% 
% \author{....lastname \thanks{...} \thanks{...} }
%                     ^------------^------------^----Do not want these spaces!
%
% a space would be appended to the last name and could cause every name on that
% line to be shifted left slightly. This is one of those "LaTeX things". For
% instance, "\textbf{A} \textbf{B}" will typeset as "A B" not "AB". To get
% "AB" then you have to do: "\textbf{A}\textbf{B}"
% \thanks is no different in this regard, so shield the last } of each \thanks
% that ends a line with a % and do not let a space in before the next \thanks.
% Spaces after \IEEEmembership other than the last one are OK (and needed) as
% you are supposed to have spaces between the names. For what it is worth,
% this is a minor point as most people would not even notice if the said evil
% space somehow managed to creep in.

\author{\IEEEauthorblockN{Priscilla~Piedra y Martín~Flores}\\
        \IEEEauthorblockA{
        Escuela de Ingeniería en Computación\\
        Instituto Tecnológico de Costa Rica. Cartago, Costa Rica
        }\\
        \small{\texttt{\{ppiedra90, mfloresg\}}\texttt{@gmail.com}}% <-this % stops a space
\thanks{Este documento fue realizado durante el curso Redes de Computadoras Avanzadas, impartido por el profesor Luis Carlos Loaiza Canet. Programa de Maestría en Computación, Instituto Tecnológico de Costa Rica. Segundo Semestre, 2017.}
}

% The paper headers
%\markboth{Journal of \LaTeX\ Class Files,~Vol.~13, No.~9, September~2014}%
%{Shell \MakeLowercase{\textit{et al.}}: Bare Demo of IEEEtran.cls for Computer Society Journals}

\markboth{Redes de Computadoras Avanzadas, Octubre 2017}%
{Shell \MakeLowercase{\textit{et al.}}: Bare Demo of IEEEtran.cls for Computer Society Journals}


% The only time the second header will appear is for the odd numbered pages
% after the title page when using the twoside option.
% 
% *** Note that you probably will NOT want to include the author's ***
% *** name in the headers of peer review papers.                   ***
% You can use \ifCLASSOPTIONpeerreview for conditional compilation here if
% you desire.



% The publisher's ID mark at the bottom of the page is less important with
% Computer Society journal papers as those publications place the marks
% outside of the main text columns and, therefore, unlike regular IEEE
% journals, the available text space is not reduced by their presence.
% If you want to put a publisher's ID mark on the page you can do it like
% this:
%\IEEEpubid{0000--0000/00\$00.00~\copyright~2014 IEEE}
% or like this to get the Computer Society new two part style.
%\IEEEpubid{\makebox[\columnwidth]{\hfill 0000--0000/00/\$00.00~\copyright~2014 IEEE}%
%\hspace{\columnsep}\makebox[\columnwidth]{Published by the IEEE Computer Society\hfill}}
% Remember, if you use this you must call \IEEEpubidadjcol in the second
% column for its text to clear the IEEEpubid mark (Computer Society jorunal
% papers don't need this extra clearance.)



% use for special paper notices
%\IEEEspecialpapernotice{(Invited Paper)}



% for Computer Society papers, we must declare the abstract and index terms
% PRIOR to the title within the \IEEEtitleabstractindextext IEEEtran
% command as these need to go into the title area created by \maketitle.
% As a general rule, do not put math, special symbols or citations
% in the abstract or keywords.
\IEEEtitleabstractindextext{%
\begin{abstract}
En este trabajo se estudia los estándares de red MPLS diseñado para unificar el servicio de transporte de datos para las redes y  DWDM que es una técnica de transmisión de señales por medio de fibra óptica que utiliza la banda C.
\end{abstract}

% Note that keywords are not normally used for peerreview papers.
%\begin{IEEEkeywords}
%Computer Society, IEEEtran, journal, \LaTeX, paper, template.
%\end{IEEEkeywords}
}


% make the title area
\maketitle


% To allow for easy dual compilation without having to reenter the
% abstract/keywords data, the \IEEEtitleabstractindextext text will
% not be used in maketitle, but will appear (i.e., to be "transported")
% here as \IEEEdisplaynontitleabstractindextext when the compsoc 
% or transmag modes are not selected <OR> if conference mode is selected 
% - because all conference papers position the abstract like regular
% papers do.
\IEEEdisplaynontitleabstractindextext
% \IEEEdisplaynontitleabstractindextext has no effect when using
% compsoc or transmag under a non-conference mode.



% For peer review papers, you can put extra information on the cover
% page as needed:
% \ifCLASSOPTIONpeerreview
% \begin{center} \bfseries EDICS Category: 3-BBND \end{center}
% \fi
%
% For peerreview papers, this IEEEtran command inserts a page break and
% creates the second title. It will be ignored for other modes.
\IEEEpeerreviewmaketitle



\IEEEraisesectionheading{\section{Introducción}\label{sec:introduction}}
% Computer Society journal (but not conference!) papers do something unusual
% with the very first section heading (almost always called "Introduction").
% They place it ABOVE the main text! IEEEtran.cls does not automatically do
% this for you, but you can achieve this effect with the provided
% \IEEEraisesectionheading{} command. Note the need to keep any \label that
% is to refer to the section immediately after \section in the above as
% \IEEEraisesectionheading puts \section within a raised box.




% The very first letter is a 2 line initial drop letter followed
% by the rest of the first word in caps (small caps for compsoc).
% 
% form to use if the first word consists of a single letter:
% \IEEEPARstart{A}{demo} file is ....
% 
% form to use if you need the single drop letter followed by
% normal text (unknown if ever used by IEEE):
% \IEEEPARstart{A}{}demo file is ....
% 
% Some journals put the first two words in caps:
% \IEEEPARstart{T}{his demo} file is ....
% 
% Here we have the typical use of a "T" for an initial drop letter
% and "HIS" in caps to complete the first word.
\IEEEPARstart{L}{as} demandas actuales en las redes se vuelven mas exigentes donde es necesario traer nuevas tecnologías que puedan soportar las redes. Una red es un conjunto de dispositivos conectados entre si que se utiliza como canales de transmisión de datos. La fibra óptica es una solución para la transmisión de datos donde los primeros sistemas eran sencillos cuyo fin era transmitir información mediante hebras de vidrio que transmitían pulsaciones de luz. A medida que se aumenta la necesidad de información es necesario aumentar el ancho de banda por lo tanto nacen técnicas como DWDM para transmitir señales y MPLS como mecanismo de transporte. 


\section {MPLS}



Multi-Protocol Label Switching o MPLS por sus siglas en inglés se ha considerado como la tecnología clave en el futuro de las grandes redes [1]. Ha ganado terreno en el área de redes ATM pues provee funcionalidades para manejo de trafico de redes por paquetes, facilidades de IP QoS y también han contribuido a mejoras en las redes VPN. 

MPLS es una propuesta para solucionar los problemas que presentan las redes como velocidad, escalabilidad, ingeniería de trafico y gestión de QoS.

Es un estándar de la IETF (Internet Engineering Task Force): RFC 3031 basada en la conmutación de marcas de Cisco inspirado en el esquma de IP switching propuesta por Nokia. 

MPLS presenta un esquema de orientación a conexiones en una red IP que no esta orientada a conexión donde se busca evitar las búsquedas en las tablas de enrutamiento durante el proceso de transferencia, buscando unificar el transporte de datos para las redes de conmutación de paquetes

MPLS es muy diferente a los protocolos por brincos tradicionales pues mediante una etiqueta (label) de tamaño fijo se puede saber el header de un paquete y la información que este lleva. 


\begin{figure*}[h]
    \center
    \includegraphics[width=15cm]{osi}
    \caption{Modelo OSI. Tomado de  \emph{Desing of Virtual Private Networks with MPLS}\cite{rexford}.}
    \label{fig:tradicitional-architecture}
\end{figure*}

MPLS se caracteriza por la conmutación rápida a nivel de la capa 2 separando la función de conmutación de la de enrutamiento, logrando una conmutación pero con información de rutas permitiendo que los equipos consuman menos recursos permitiendo una implementación de ingeniería de trafico. MPLS se puede decir  que trabaja en la capa 2.5[1] segun el modelo OSI donde, la capa 2 cubre protocolos como el Ethernet y SONET[1] los cuales llevan paquetes IP pero solo entre LANS o WANs punto a punto. La capa 3 cubre direccionamiento profundo en internet y enrutamiento usando protocolos IP. MPLS se situa entre las capas 2 y 3 como se ha mencionado anteriormente proveyendo valor agregado para el transporte de datos entre la red.

MPLS tiene una parte de control encargada de las decisiones de enrutamiento y reenvió coordinadas donde se construye una tabla con etiquetas la cual es consultada para saber el flujo de los datos. Todos los paquetes que llevan la misma etiqueta forman un grupo llamado FEC o Fowarding Equivalent Class.

En este protocolo el único router que tiene que hacer funciones de enrutamiento es el que esta de primero el cual se encarga de etiquetar cada paquete.


Los siguientes pasos deben ser tomados por un paquete para viajar a través de una red MPLS:

\begin{itemize}
  \item Creación de la etiqueta y la distribución. 
  \item Creación de la tabla en cada router.
  \item Creación del flujo de intercambio de etiquetas.
  \item Reenvío de paquetes.
\end{itemize}
\subsection{Cabeceras}


\begin{figure*}[h]
    \center
    \includegraphics[width=15cm]{cabecera}
    \caption{Cabecera MPLS. Tomado de  \emph{Desing of Virtual Private Networks with MPLS}\cite{rexford}.}
    \label{fig:tradicitional-architecture}
\end{figure*}



Las etiquetas que usa MPLS van en el nivel 2 y 3[2]. En la figura anterior de tiene que el campo de label es la etiqueta, los tres bits de exp son reservados, el bit S es usado para el apilameinto mientras que el campo TTL funciona para poder mantener la funcionalidad TTL del IP. El TTL de un paquete se copia en la cabecera (header) y el mismo se va decrementando en cada router.

Cuando un router configurado con MPLS recibe un paquete, este puede realizar tres diferentes operaciones: hacer push a la etiqueta en una pila, pop de la etiqueta de la pila o bien cambiar la etiqueta en la cima de la pila.  

Las cabeceras funcionan de la siguiente manera: Si hay cuatro routers bajo el protocolo MPLS el primero de ellos va a colocar una etiqueta en el paquete y los demás la van a ir cambiando conforme el paquete es conmutado, si por ejemplo para ir del router 3 al 2 hubieran tres routers mas en otra red MPLS seria necesario que esos tres routers colocaran otra etiqueta para permitir el paso del paquete dejando la primer etiqueta intacta. A esto se le conoce como label stack o apilamiento de etiquetas. El bit S es utilizado para saber cual es la primer etiqueta en la pila.


\subsection{Label Switching}

Tradicionalmente, existen dos enfoques diferentes para el reenvío de paquetes, cada mapeo hacia una estructura específica de la tabla de reenvío. Se les llama reenvío por dirección de red y conmutación de etiquetas.


Un enfoque sencillo es el reenvío por dirección de red, o sea por IP. Cuando un paquete llega a un enrutador analiza el destino en el header del paquete y lo busca la tabla de ruteo, la cual tiene una estructura simple de 2 columnas donde cada fila guarda el destino a la interfaz de salida por el cual el paquete debe ser reenviado. 

\begin{figure*}[h]
    \center
    \includegraphics[width=15cm]{label-fowarding}
    \caption{Una red basada en intercambio de etiquetas donde se muestra la tabla de reenvíos. Tomado de  \emph{Desing of Virtual Private Networks with MPLS}\cite{rexford}.}
    \label{fig:tradicitional-architecture}
\end{figure*}

La conmutación de etiquetas (label switching) representa un enfoque alternativo (MPLS).Esencialmente, mientras que el reenvío por dirección de red requiere que la salida se elija en base al destino del paquete, la conmutación de etiquetas requiere que dicha interfaz sea elegida en base al flujo al que pertenece el paquete. Un flujo corresponde a una instancia de transmisión, es decir, un conjunto de paquetes, desde una fuente hasta un destino y se identifica mediante una etiqueta (label) adjunta a cada paquete del flujo. Un label o etiqueta es un identificador de tamaño fijo no estructurado que puede ser usado para asistir en el proceso de envío, normalmente son locales a un solo dato y no tienen un valor global significativo.

Un router label switching es cualquier dispositivo que soporte control estándar de IP y el componente de intercambio de etiquetas.  Cuando el paquete llega a un router, el router extrae la etiqueta del header, busca el valor de la etiqueta en la tabla de reenvió y su interfaz de egreso por el cual el paquete debería ser enviado y una nueva etiqueta es agregada al paquete. 

El label switching es una forma avanzada de reenvió que reemplaza las direcciones largas con una etiqueta mas eficiente con un algoritmo de intercambio. Hay tres grandes diferencias entre el enrutamiento de etiquetas y el enrutamiento convencional[1]:

\begin{itemize}
  \item Análisis de encabezado IP: En el enrutamiento convencional ocurre en cada nodo mientras que en el de etiquetas ocurre solo una vez cuando la misma es asignada. 
  \item Soporte Unicast y Multicast: Para el enrutamiento convencional se requiere de algoritmos complejos de re-envío, para el enrutamiento por etiquetas solo se ocupa un algoritmo de reenvío.
  \item Decisiones de ruteo: Para el enrutamiento convencional esta basado solo en dirección y para el enrutamiento de etiquetas el mismo puede se basado en cualquier numero de parámetros como QoS, o miembros de VPN.
\end{itemize}



Una red basada en intercambio de etiquetas sirve con el mismo propósito de cualquier red convencional entregando trafico a uno o mas destinos. Agregar las etiquetas complementa el enrutamiento básico pero nunca lo reemplaza, al contrario trae retos a los protocolos de reenvió[2] , esto pues antes de transmitir un nuevo flujo a través de una ruta una nueva etiqueta debe ser generada y asignada para cada tramo de la ruta. Para evitar elegir etiquetas ya usadas existe una regla donde las mismas no pueden ser idénticas para cada interfaz dentro de un router pero, se pueden repetir fuera de los mismos. Por esta razón es que las etiquetas tienen que cambiar en cada salto. Un ejemplo de una tabla de reenvió es el que se muestra en la siguiente figura.

\begin{figure*}[h]
    \center
    \includegraphics[width=15cm]{tabla}
    \caption{Estructura de una tabla de envios basado en intercambio de etiquetas. Tomado de \emph{Desing of Virtual Private Networks with MPLS}\cite{rexford}.}
    \label{fig:tradicitional-architecture}
\end{figure*}


El label switching no es único para MPLS, otros protocolos como ATM y Frame Relay tradicionalmente adoptan el mismo mecanismo de reenvío donde inicialmente la razón para preferir el cambio de etiquetas era el rendimiento ya que, es mas sencillo buscar una etiqueta en la tabla de reenvío que buscar una dirección IP.
Además de que las etiquetas pueden tomar valores en un rango mucho más pequeño que las direcciones IP, los valores de la etiqueta se pueden buscar exactamente (de forma eficiente, como por ejemplo implementando tablas hash) mientras que las direcciones IP deben ser buscadas de forma exacta. Sin embargo, los routers modernos utilizan
hardware y datos eficientes para implementar sus tablas de reenvío, obteniendo rendimiento en la búsqueda. 


Algunas de las ventajas del reenvió de etiquetas es que las mismas pueden ser realizadas por conmutadores de baja capacidad además de que la asignación de FEC  puede hacerse en base a información disponible del paquete como  por ejemplo el puerto de origen o el enrutador de entrada. También  los criterios para asignar el FEC pueden volverse complejos sin necesidad de afectar a los enrutadores. En ingeniería de trafico, el uso de la etiqueta es fundamental pues esta es usada para indicar el flujo de los paquetes de forma prioritaria. 
También cabe destacar que algunos enrutadores pueden analizar la cabecera del paquete para determinar la prioridad o la clase del servicio y no solo el salto que este debe dar. 


%\begin{figure*}[h]
%    \center
%    \includegraphics[width=15cm]{sdn-history}
%    \caption{Desarrollos selectos en el campo de redes programables en los últimos 20 años. Tomado de \emph{The Road to SDN}\cite{rexford}.}
%    \label{fig:tradicitional-architecture}
%\end{figure*}


\section{DWDM}
El crecimiento explosivo del Internet y de aplicaciones basadas en el protocolo de Internet (IP) tales como voz, video y redes de área de almacenamiento (SAN, por sus siglas en inglés) ha impulsado las demandas de ancho de banda de muchas organizaciones. Con las velocidades de redes LAN en el rango de los 10Mbps a los 10Gbps, y la calidad de servicio (QoS) jugando un rol importante en la entrega de estos datos, debe de existir un alternativa a servicios WAN y LAN para conectar redes de área metropolitana (MAN). Las conexiones de red que tradicionalmente llevaban velocidades de datos \textsc{T1} y \textsc{T3} ahora requieren de canal de fibra, \emph{Enterprise System Connection} (ESCON), Gigabit Ethernet y 10 Gigabit Ethernet para satisfacer la demanda. El aumento en la demando, junto con los avances en la tecnología óptica han aumentado drásticamente la capacidad y reducción de costos, haciendo más atractivo a los proveedores de servicios ofrecer servicios de red basedos en fibra para el mercado metropolitano. 

Las redes metropolitanas basadas en fibra abordan necesidades de negociones en tres áreas:
\begin{enumerate}
    \item Redes de datos y migración: tecnologías de red ópticas ofrecen muchas velocidades de datos y conexiones que soportan una variedad de tecnologías de red, tales como IP, \emph{synchronous optical networks} (SONET), ATM, y TDM. Las redes pueden consolidar múltiples longitudes de onda de tráfico a una sola fibra para proporcionar multiservicio de transporte y facilitar la migración de tecnologías tradicionales de redes eléctricas en un trasponer óptico común.
    \item Recuperación de desastres y continuidad de negocio: tener un centro de respaldo de datos es una consideración primaria para muchas compañías grandes de hoy. Las redes ópticas metropolitanas proporcionan transporte rápido de \emph{campus} a \emph{campus} con redundancia. Soluciones de recuperación de desastres en tiempo real, tales como reflejo sincrónico (\emph{synchronous mirroring}), garantizan que los datos de misión crítica estan sincronizados de forma segura y remota para evitar pérdida de datos en el caso de un evento desastroso.
    \item Consolidación del almacenamiento: cuando el almacenamiento conectado a la red (NAS) se integra con aplicaciones SAN se provee consolidación de almacenamiento basado en IP y compartición de archivos. Las redes ópticas metropolitanas facilitan no solo la implementación del almacenamiento, sino también la extensión del almacenamiento más allá de un solo centro de datos. 
   \item Fibra residencial: conforma la fibra se hace más accesible en costos, ha empezado a penetrar en el mercado residencial.
\end{enumerate}


Las tres principales tecnologías ópticas hoy en día son:
\begin{enumerate}
    \item \emph{Synchronous Optional NET} (SONET)
    \item \emph{Dense wavelength division multiplexing} (DWDM)
    \item \emph{Dynamic packet transport} (DPT)
\end{enumerate}

Las tres convierten señales eléctricas en luz y viceversa. Los \emph{Fiber-Optic Transmission Systems} (FOTS) hacen la conversión. Las señales de fibra óptica no son suceptibles a la interferencia eléctrica. Las señales se pueden transmitir en distancias largas y pueden enviar más información que los medio de trasporte eléctricos tradicionales. La combinación de estos beneficiones provee costos más bajo que en mecanismos de transporte de datos eléctricos tradicionales.

DWDM se basa en la premisa que las señales ópticas de diferente longitud de onda no interfieren entre sí. Multiplexación por división de longitud de onda (WDM) difiere de tecnologías de multiplexación de división de tiempo (como SONET) de la siguiente forma:
\begin{itemize}
    \item \emph{Time-Division Multiplexing} (TDM) emplea una sola longitud de one a través de la fibra. Los datos son divididos en canales del tal form que múltiples canales pueden viajar a través de una sola fibra.
    \item WDM emplea múltimples longitudes de onda (lambdas) por fibra, lo que permite que múltiples canales por fibra (hasta 160). Cada lambda puede incluir múltiples canales TDM.
\end{itemize}

DWDM ofrece escalabilidad por sobre las tecnologías tradiciones de TDM. Esto por los datos pueden viajar considerablemente más lejos a través de DWDM que en TDM tradicionales (120Km vs 40 Km), se necesitan menos repetidoras. DWDM también permite una mayor capacidad a través de las fibras de largo recorrido, así como un rápido provisionamiento en las redes metropolitanas. DWDM metropolitanos necesitan ser baratos y simples de gestionar. Deben también ser independientes de las tasas de bits y protocolo como transporte y proporcionar de 16 a 32 canales por fibra. Los nodos DWDM se unen unos con otros en un patrón de anillo usando multiplexadores óptidos \emph{add-drop}, los cuales agregan y evitan tráfico en cada sitio remoto, y todo el tráfico se dirigue a un sitio central.

\subsection{Fundamentos de la fibra}
Los dos tipo de fibra óptica son multimodo y monomodo. Con fibra multimodo, la luz se propaga en forma de múltiples longitudes de onda, cada una tomando una ruta ligeramente diferente. La fibra multimodo es usada principalmente en sistemas de corta transmisión (menos de 2Km). La fibra monomodo tien un solo modo en el cual la luz se propaga. La fibra monomodo se utiliza usualmente en largas distancias y en aplicaciones de alto ancho de banda.


\begin{figure*}[h]
    \center
    \includegraphics[width=8cm]{fibra-multimodo}
    \caption{Modelo OSI. Tomado de  \emph{Desing of Virtual Private Networks with MPLS}\cite{rexford}.}
    \label{fig:fibra-multimodo}
\end{figure*}

\subsection{Criterio de diseño}
A medida que viajan por las fibras, las señales ópticas están sujetas a deterioro en la forma de atenuación, dispersión y no linealidades, distorsión cromática y distorsión del modo de polatización. Estos factores limitan la distancia y el ancho de banda de las señales ópticas. 

\subsubsection{Multiplexación}
Multiplexación es el proceso de combinar múltiples señales en un solo cable, fibra o enlace. Con TDM, se introducen señales de baja velocidad, se asignan ranuras de tiempo y se colocan en una salida en serie de velocidad más alta. En el extremo receptor, las señales son reconstruidas.

%% Posible imagen aqui

Una de las propiedades de la luz es que las ondas de luz de diferente longitud de onda no interfieren unas con otras dentro de un medio. Debido a esta propiedad, cada longitud de onda de luz puede representar un canal diferente de información. Al combinar pulsos de luz de diferente longitud de onda, muchos canals pueden ser transmitidos sobre una sola fibra simultaneamente. A esto se le llama Multiplexación por división de longitud de onda (WDM). WDM es una variante de multiplexación de división de frecuencia (FDM).

\subsubsection{Optical Ethernet}
Gigabit Ethernet es la forma más simple y barata de transporte óptico. Optical Ethernet utiliza un dispositivo llamado convertidores de interface gigabit (GBIC), el cual se conecta a un puerto de un \emph{switch} y convierte un flujo Ethernet en una señal óptica.

\subsubsection{CWDM}
\emph{Coarse Wavelength Division Multiplexing} (CWDM) utiliza pares de GBICs de longitud de onda específicos para combinar hasta ocho señales ópticas en una sola fibra. Cada para de \emph{switches} está equipado con uno o más pares de GBICs. Cada par de GBICs se sintoniza con una frecuencia específica que permita que el \emph{switch} añada (mux) o ``arrebate'' (demux) un rayo de luz (un flujo de datos).

CWDM puede ser implementado como un anillo o punto-a-punto. Un gran inconveniente de CDWM es que no puede ser amplificado, lo que hace que esta solución tenga una distancia limitada. La regla que se sugiere es una distancia de 80 Km para punto-a-punto o una circunferencia de anillo de 30 Km.

\subsubsection{DWDM}
Usa el mismo esquema de multiplexación qeu CWDM. Sin embargo, las señales DWDM, están mucho más cerca unas de otras, permitiendo a los sistemas DWDM multiplexar hasta 32 señales en una sola fibra. 

Las señales DWDM pueden ser amplificadas, haciendo este sistema el ideal para centros de respaldo de datos o para organizaciones con sucursales geograficamente dispersas. Con amplificación, las señales DWDM pueden ser transmitidas hasta 250Km.

% needed in second column of first page if using \IEEEpubid
%\IEEEpubidadjcol


% An example of a floating figure using the graphicx package.
% Note that \label must occur AFTER (or within) \caption.
% For figures, \caption should occur after the \includegraphics.
% Note that IEEEtran v1.7 and later has special internal code that
% is designed to preserve the operation of \label within \caption
% even when the captionsoff option is in effect. However, because
% of issues like this, it may be the safest practice to put all your
% \label just after \caption rather than within \caption{}.
%
% Reminder: the "draftcls" or "draftclsnofoot", not "draft", class
% option should be used if it is desired that the figures are to be
% displayed while in draft mode.
%
%\begin{figure}[!t]
%\centering
%\includegraphics[width=2.5in]{myfigure}
% where an .eps filename suffix will be assumed under latex, 
% and a .pdf suffix will be assumed for pdflatex; or what has been declared
% via \DeclareGraphicsExtensions.
%\caption{Simulation results for the network.}
%\label{fig_sim}
%\end{figure}

% Note that IEEE typically puts floats only at the top, even when this
% results in a large percentage of a column being occupied by floats.
% However, the Computer Society has been known to put floats at the bottom.


% An example of a double column floating figure using two subfigures.
% (The subfig.sty package must be loaded for this to work.)
% The subfigure \label commands are set within each subfloat command,
% and the \label for the overall figure must come after \caption.
% \hfil is used as a separator to get equal spacing.
% Watch out that the combined width of all the subfigures on a 
% line do not exceed the text width or a line break will occur.
%
%\begin{figure*}[!t]
%\centering
%\subfloat[Case I]{\includegraphics[width=2.5in]{box}%
%\label{fig_first_case}}
%\hfil
%\subfloat[Case II]{\includegraphics[width=2.5in]{box}%
%\label{fig_second_case}}
%\caption{Simulation results for the network.}
%\label{fig_sim}
%\end{figure*}
%
% Note that often IEEE papers with subfigures do not employ subfigure
% captions (using the optional argument to \subfloat[]), but instead will
% reference/describe all of them (a), (b), etc., within the main caption.
% Be aware that for subfig.sty to generate the (a), (b), etc., subfigure
% labels, the optional argument to \subfloat must be present. If a
% subcaption is not desired, just leave its contents blank,
% e.g., \subfloat[].


% An example of a floating table. Note that, for IEEE style tables, the
% \caption command should come BEFORE the table and, given that table
% captions serve much like titles, are usually capitalized except for words
% such as a, an, and, as, at, but, by, for, in, nor, of, on, or, the, to
% and up, which are usually not capitalized unless they are the first or
% last word of the caption. Table text will default to \footnotesize as
% IEEE normally uses this smaller font for tables.
% The \label must come after \caption as always.
%
%\begin{table}[!t]
%% increase table row spacing, adjust to taste
%\renewcommand{\arraystretch}{1.3}
% if using array.sty, it might be a good idea to tweak the value of
% \extrarowheight as needed to properly center the text within the cells
%\caption{An Example of a Table}
%\label{table_example}
%\centering
%% Some packages, such as MDW tools, offer better commands for making tables
%% than the plain LaTeX2e tabular which is used here.
%\begin{tabular}{|c||c|}
%\hline
%One & Two\\
%\hline
%Three & Four\\
%\hline
%\end{tabular}
%\end{table}


% Note that the IEEE does not put floats in the very first column
% - or typically anywhere on the first page for that matter. Also,
% in-text middle ("here") positioning is typically not used, but it
% is allowed and encouraged for Computer Society conferences (but
% not Computer Society journals). Most IEEE journals/conferences use
% top floats exclusively. 
% Note that, LaTeX2e, unlike IEEE journals/conferences, places
% footnotes above bottom floats. This can be corrected via the
% \fnbelowfloat command of the stfloats package.




\section{Conclusión}

 MPLS fue diseñado para unificar el servicio de transporte de datos para las redes .MPLS hoy en dia es una solución de mayor alcance, donde se necesitan reducir costos, aumentar la productividad y poder soportar mas aplicaciones en la nube tomando en cuenta la seguridad llevan a muchas empresas actualmente a implementar MPLS. MPLS tiene la capacidad de integrar voz, vídeo y datos en una plataforma común con calidad brindando además rendimiento y escalabilidad de servicios, esto pues MPLS permite empaquetar mas datos en el ancho de banda disponible y disminuir requerimientos a nivel de router.
 
 DWDM es una técnica de transmisión de señales por medio de fibra óptica que utiliza la banda C que introduce mas longitudes de onda distintas en cada fibra y por esta razón es actualmente una de las técnicas mas implementadas , permitiendo fabricación a gran escala, disminución de costos, mejorar en la transmisión de fibra, implementación de diferentes técnicas de fabricación e implementación en redes de larga distancia.  





% if have a single appendix:
%\appendix[Proof of the Zonklar Equations]
% or
%\appendix  % for no appendix heading
% do not use \section anymore after \appendix, only \section*
% is possibly needed

% use appendices with more than one appendix
% then use \section to start each appendix
% you must declare a \section before using any
% \subsection or using \label (\appendices by itself
% starts a section numbered zero.)
%


%\appendices
%\section{Proof of the First Zonklar Equation}
%Appendix one text goes here.
%
%\section{}
%Appendix two text goes here.


% use section* for acknowledgment
%\ifCLASSOPTIONcompsoc
%  % The Computer Society usually uses the plural form
%  \section*{Acknowledgments}
%\else
%  % regular IEEE prefers the singular form
%  \section*{Acknowledgment}
%\fi


%The authors would like to thank...


% Can use something like this to put references on a page
% by themselves when using endfloat and the captionsoff option.
\ifCLASSOPTIONcaptionsoff
  \newpage
\fi



% trigger a \newpage just before the given reference
% number - used to balance the columns on the last page
% adjust value as needed - may need to be readjusted if
% the document is modified later
%\IEEEtriggeratref{8}
% The "triggered" command can be changed if desired:
%\IEEEtriggercmd{\enlargethispage{-5in}}

% references section

% can use a bibliography generated by BibTeX as a .bbl file
% BibTeX documentation can be easily obtained at:
% http://www.ctan.org/tex-archive/biblio/bibtex/contrib/doc/
% The IEEEtran BibTeX style support page is at:
% http://www.michaelshell.org/tex/ieeetran/bibtex/
%\bibliographystyle{IEEEtran}
% argument is your BibTeX string definitions and bibliography database(s)
%\bibliography{IEEEabrv,../bib/paper}
%
% <OR> manually copy in the resultant .bbl file
% set second argument of \begin to the number of references
% (used to reserve space for the reference number labels box)
\begin{thebibliography}{1}

\bibitem{bombal}
CISCO. \emph{MPLS Concepts}. \hskip 1em plus 0.5em minus 0.4em\relax CISCO, Implementing CISCO MPLS. 2002.

\bibitem{cittadini}
L.~Cittadini. \emph{Desing of Virtual Private Networks with MPLS}. \hskip 1em plus 0.5em minus 0.4em\relax University of Cambridge.

P.~Della Maggiora, N.~Anderson y J.~Doherty. \emph{Cisco Networking Simplified}.\hskip 1em plus 0.5em minus 0.4em\relax Segunda edición. Cisco Press. 2007.

\bibitem{cittadini}
E.~Rosen, A. ~Viswanathan, R.~Callon.\emph{ Multiprotocol Label Switching
Architecture )}. \hskip 1em plus 0.5em minus 0.4em\relax . RFC 3031, 2001 

\end{thebibliography}

% biography section
% 
% If you have an EPS/PDF photo (graphicx package needed) extra braces are
% needed around the contents of the optional argument to biography to prevent
% the LaTeX parser from getting confused when it sees the complicated
% \includegraphics command within an optional argument. (You could create
% your own custom macro containing the \includegraphics command to make things
% simpler here.)
%\begin{IEEEbiography}[{\includegraphics[width=1in,height=1.25in,clip,keepaspectratio]{mshell}}]{Michael Shell}
% or if you just want to reserve a space for a photo:

\begin{IEEEbiography}[{\includegraphics[width=1in,height=1.25in,clip,keepaspectratio]{priscilla-piedra}}]{Priscilla Piedra}
es Ingeniera de Computación del Tecnologíco de Costa Rica. Actualmente es estudiante del programa de Maestría en Ciencas de la Computación en la misma universidad. Sus principales intereses son: \emph{cloud computing} y automatización. 
\end{IEEEbiography}

% if you will not have a photo at all:
\begin{IEEEbiography}[{\includegraphics[width=1in,height=1.25in,clip,keepaspectratio]{martin-flores}}]{Martín Flores}
es Ingeniero en Informática de la Universidad Nacional. Actualmente, realiza sus estudios de Maestría en Ciencias de la Computación del Tecnológico de Costa Rica. Sus principales intereses son: lenguajes de programación, ingeniería de software y \emph{DevOps}.
\end{IEEEbiography}

% insert where needed to balance the two columns on the last page with
% biographies
%\newpage

%\begin{IEEEbiographynophoto}{Jane Doe}
%Biography text here.
%\end{IEEEbiographynophoto}

% You can push biographies down or up by placing
% a \vfill before or after them. The appropriate
% use of \vfill depends on what kind of text is
% on the last page and whether or not the columns
% are being equalized.

\vfill

% Can be used to pull up biographies so that the bottom of the last one
% is flush with the other column.
%\enlargethispage{-5in}



% that's all folks
\end{document}


